\documentclass[10pt, a4paper, twocolumn]{article}

% Packages
\usepackage[utf8]{inputenc}
\usepackage[T1]{fontenc}
\usepackage{geometry}
\usepackage{graphicx}
\usepackage{amsmath, amssymb, amsfonts}
\usepackage{hyperref}
\usepackage{natbib}
\usepackage{float}
\usepackage{booktabs}
\usepackage{caption}
\usepackage{titlesec}

% Geometry settings for two-column paper
\geometry{
    margin=0.75in,
    columnsep=0.25in
}

% Formatting
\titleformat{\section}{\large\bfseries}{\thesection}{1em}{}
\titleformat{\subsection}{\bfseries}{\thesubsection}{1em}{}

% Title parameters
\title{\textbf{Deep Q-Learning for No-Limit Texas Hold'em: \\ Architecture and Convergence Analysis}}
\author{
    Kacper Duda, Marek Dzierżawa, Kacper Karabasz\\
    \textit{AGH University of Krakow} 
}
\date{\today}

\begin{document}

\maketitle

\begin{abstract}
    We present a Reinforcement Learning approach to solving No-Limit Texas Hold'em (NLHE), a high-dimensional extensive-form game with imperfect information. Our solution implements a Deep Q-Network (DQN) agent utilizing a dense reward signal based on normalized chip accumulation. We engineer a compact 134-dimensional feature vector encapsulating hole cards, board state, and pot odds. The model architecture features a split-head Multi-Layer Perceptron (MLP) for simultaneous discrete action selection and continuous bet-sizing. We evaluate the agent's convergence properties over 2,000 episodes, demonstrating early-stage strategy acquisition against stochastic policies.
\end{abstract}

\section{Introduction}
No-Limit Texas Hold'em presents a significant challenge for conventional game-theoretic solvers due to its vast decision tree and hidden information states ($\approx 10^{161}$ decision points). While Counterfactual Regret Minimization (CFR) remains the state-of-the-art for Nash Equilibrium approximation in Poker, Deep Reinforcement Learning (DRL) offers a scalable alternative by approximating the optimal value function $Q^*(s,a)$ directly from self-play or gameplay experience.

This work implements a Model-Free RL agent using Deep Q-Learning (DQN) with Experience Replay. We focus on the feature engineering required to represent the poker game state efficiently for a feed-forward neural network and analyze the stability of learning in a 3-player environment.

\section{Methodology}

\subsection{State Representation}
The partial observation $o_t$ at time $t$ is mapped to a feature vector $\phi(o_t) \in \mathbb{R}^{134}$. The state space $S$ is composed of:

\begin{itemize}
    \item \textbf{Card Embeddings}: Hero's hand $H \in \{0,1\}^{52}$ and Community cards $C \in \{0,1\}^{52}$ represented via one-hot encoding.
    \item \textbf{Game Context}: Normalized stack sizes $s_i$, current bets $b_i$, andpot size $P$, such that all values lie approximately in $[0, 1]$.
    \item \textbf{Heuristic Features}: To accelerate training, we inject domain knowledge via pre-computed hand strength metrics $E(H, C) \in \mathbb{R}^{14}$ (bucketed hand rank + normalized high cards).
\end{itemize}

\subsection{Network Architecture}
We employ a dense Multi-Layer Perceptron (MLP) parameterized by $\theta$. The network architecture (Fig. \ref{fig:arch}) consists of:
\begin{align}
    h_1 &= \text{ReLU}(W_1 x + b_1), & W_1 \in \mathbb{R}^{512 \times 134} \nonumber \\
    h_2 &= \text{ReLU}(W_2 h_1 + b_2), & W_2 \in \mathbb{R}^{512 \times 512} \nonumber \\
    h_3 &= \text{ReLU}(W_3 h_2 + b_3), & W_3 \in \mathbb{R}^{256 \times 512} \nonumber 
\end{align}

The final layer splits into two heads:
\begin{enumerate}
    \item \textbf{Action Value Head} ($Q(s, \cdot) \in \mathbb{R}^3$): Linearly estimates Q-values for discrete actions $A = \{\text{Fold, Call, Raise}\}$.
    \item \textbf{Sizing Head} ($\sigma(s) \in [0, 1]$): A Sigmoid unit determining the raise magnitude as a fraction of the legal betting interval (min-raise to all-in).
\end{enumerate}

\begin{figure}[h]
    \centering
    \includegraphics[width=\linewidth]{model_architecture.png}
    \caption{PokerNet Architecture. The network maps the 134-dim state vector to action-values and a continuous bet-sizing parameter.}
    \label{fig:arch}
\end{figure}

\subsection{Training Algorithm}
The agent minimizes the temporal difference error using the Huber Loss to ensure robustness against outliers in variance-heavy poker rewards:
\begin{equation}
    \mathcal{L}(\theta) = \mathbb{E}_{(s, a, r, s') \sim \mathcal{D}} \left[ \delta_\theta^2 \right]
\end{equation}
where $\delta_\theta = r + \gamma \max_{a'} Q(s', a'; \theta^-) - Q(s, a; \theta)$.

\begin{table}[h]
\centering
\caption{Hyperparameters}
\label{tab:params}
\begin{tabular}{ll}
\toprule
Parameter & Value \\
\midrule
Batch Size & 128 \\
Learning Rate ($\alpha$) & $1 \times 10^{-4}$ (Adam) \\
Discount Factor ($\gamma$) & 0.99 \\
Target Update Period & 500 steps \\
Replay Buffer Size & $10^5$ transitions \\
$\epsilon$-decay strategy & Exp. decay ($1.0 \to 0.05$) \\
\bottomrule
\end{tabular}
\end{table}

\section{Experiments and Results}

\subsection{Convergence Analysis}
Training was conducted over 2,000 episodes. The dense reward function is defined as the normalized net profit at the hand's conclusion.

\begin{figure}[h]
    \centering
    \includegraphics[width=\linewidth]{learning_curve.png}
    \caption{Training progression. Blue (dashed): Win Rate (\%). Red (solid): Moving average of Normalized Reward.}
    \label{fig:learning}
\end{figure}

Fig. \ref{fig:learning} illustrates the training dynamics. The high variance is intrinsic to poker (luck factor). However, the upward trend in Average Reward indicates the policy $\pi_\theta$ is improving over the random baseline. The win rate oscillation suggests the agent is shifting between aggressive and passive strategies as it explores the state space.

\subsection{Visual Verification}
A custom UI (Fig. \ref{fig:ui}) validates the internal state tracking and decision-making process.

\begin{figure}[h]
    \centering
    \includegraphics[width=\linewidth]{ui_screenshot.png}
    \caption{Environment Visualization.}
    \label{fig:ui}
\end{figure}

\section{Conclusion}
The proposed DQN architecture successfully parses the complex poker state space. Feature engineering significantly aids the MLP in recognizing hand strength. Future work addresses the discrete-continuous hybrid action space proper handling (e.g., via Parameterized Action Space Q-Network or PPO) and migrating to Self-Play (SP) to minimize exploitability rather than maximizing profit against fixed policies.

\bibliographystyle{abbrvnat}
\bibliography{references}

\end{document}
